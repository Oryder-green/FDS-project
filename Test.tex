% Options for packages loaded elsewhere
\PassOptionsToPackage{unicode}{hyperref}
\PassOptionsToPackage{hyphens}{url}
%
\documentclass[
]{article}
\usepackage{amsmath,amssymb}
\usepackage{lmodern}
\usepackage{iftex}
\ifPDFTeX
  \usepackage[T1]{fontenc}
  \usepackage[utf8]{inputenc}
  \usepackage{textcomp} % provide euro and other symbols
\else % if luatex or xetex
  \usepackage{unicode-math}
  \defaultfontfeatures{Scale=MatchLowercase}
  \defaultfontfeatures[\rmfamily]{Ligatures=TeX,Scale=1}
\fi
% Use upquote if available, for straight quotes in verbatim environments
\IfFileExists{upquote.sty}{\usepackage{upquote}}{}
\IfFileExists{microtype.sty}{% use microtype if available
  \usepackage[]{microtype}
  \UseMicrotypeSet[protrusion]{basicmath} % disable protrusion for tt fonts
}{}
\makeatletter
\@ifundefined{KOMAClassName}{% if non-KOMA class
  \IfFileExists{parskip.sty}{%
    \usepackage{parskip}
  }{% else
    \setlength{\parindent}{0pt}
    \setlength{\parskip}{6pt plus 2pt minus 1pt}}
}{% if KOMA class
  \KOMAoptions{parskip=half}}
\makeatother
\usepackage{xcolor}
\usepackage[margin=1in]{geometry}
\usepackage{graphicx}
\makeatletter
\def\maxwidth{\ifdim\Gin@nat@width>\linewidth\linewidth\else\Gin@nat@width\fi}
\def\maxheight{\ifdim\Gin@nat@height>\textheight\textheight\else\Gin@nat@height\fi}
\makeatother
% Scale images if necessary, so that they will not overflow the page
% margins by default, and it is still possible to overwrite the defaults
% using explicit options in \includegraphics[width, height, ...]{}
\setkeys{Gin}{width=\maxwidth,height=\maxheight,keepaspectratio}
% Set default figure placement to htbp
\makeatletter
\def\fps@figure{htbp}
\makeatother
\setlength{\emergencystretch}{3em} % prevent overfull lines
\providecommand{\tightlist}{%
  \setlength{\itemsep}{0pt}\setlength{\parskip}{0pt}}
\setcounter{secnumdepth}{-\maxdimen} % remove section numbering
\ifLuaTeX
  \usepackage{selnolig}  % disable illegal ligatures
\fi
\IfFileExists{bookmark.sty}{\usepackage{bookmark}}{\usepackage{hyperref}}
\IfFileExists{xurl.sty}{\usepackage{xurl}}{} % add URL line breaks if available
\urlstyle{same} % disable monospaced font for URLs
\hypersetup{
  pdftitle={Draft},
  pdfauthor={Oliver Ryder-Green},
  hidelinks,
  pdfcreator={LaTeX via pandoc}}

\title{Draft}
\author{Oliver Ryder-Green}
\date{2022-12-03}

\begin{document}
\maketitle

\clearpage

\section{Introduction}

Flight delays are an inconvenience that almost all aviation passengers
will experience at some point in their travels. Yet the burden of flight
delays is not the same for all passengers. In particular, US passengers
are not entitled to compensation for
delays\footnote{source:www.transportation.gov}. Yet, between 2013 and
2022, approximately one in every five flights from US airports was
delayed by at least 15 minutes\footnote{source:www.bts.gov}. With more
than 10 million scheduled passenger flights in the US each
year\footnote{source:www.faa.gov}, the cost to passengers of flight
delays is substantial. Indeed, the
\textit{Federal Aviation Administration} estimates that flight delays in
the US from 2016 to 2019 cost passengers US\$62.6billion in total. Short
of relying on airlines to inform them of expected delays, there is
little that US passengers can do to reliably avoid flight delays.
Therefore, I apply the classification methods discussed in class to
determine which factors inform flight departure delays for domestic
flights in the US.\\

Data from the \textit{Bureau of Transportation Statistics} illustrates
the prevalence of domestic flight delays. Among all US carriers, between
15--25\% of departures were delayed from 2010 to 2022. Among the top US
carriers\footnote{as measured by total number of flights serviced in 2010--2022.},
the proportion of delayed flights is persistently higher than average.
Evidently, some US carriers exhibit fewer than average flight delays
(e.g., Delta Airlines), but top US carriers tend to demonstrate more
frequent flight delays than the industry as a whole.\\

For US passengers, the fact that top US carriers experience more
frequent departure delays may be of interest in trying to avoid delays.
That said, more frequent delays at top airlines do not necessarily imply
more severe (i.e., costly) delays for passengers. The
\textit{Bureau of Transportation Statistics} data shows that, among all
US carriers, mean departure delay lengths were between 7 and 17 minutes
on average from 2010 to 2022. Unfortunately, top US carriers again
appear to perform worse than the industry as a whole. Without exception,
top US carriers exhibit longer-than-average delays at some point in the
period.\\

The \textit{Bureau of Transportation Statistics} data also highlights
that the frequency of delays varies by origin airport. In line with the
above, around one in every five flights from a US airport is delayed.
There are clearly some airports that persistently experience more
frequent delays, over 50\% of all flights in some cases, and some
airports that experience few or no delays.

The task of anticipating delays is extremely difficult for passengers.
Many factor are hard to observe or nearly impossible to predict. That
said, the data above suggests that some readily observable features may
be useful for passengers trying to avoid delays. For instance, if
passengers face a choice of carriers, they may be better able to avoid
costly delays by choosing those that exhibit less frequent and shorter
delays. The aim of this analysis is to identify such features that
passengers might use to anticipate delays.

\section{Data}

To identify factors that inform whether a flight is delayed on
departure, I use data from the
\textit{Bureau of Transportation Statistics' Airline On-Time Performance Data}\footnote{www.transtats.bts.gov/}
for January, March, September, and December in 2016, 2017, and 2018,
respectively. The flight data contains 8,777 observations on US domestic
flights and 21 features, such as the flight date, origin airport,
carrier, destination, distance, and other flight level characteristics.
I combine this data with weather data from
\textit{Weather Underground}\footnote{www.wunderground.com}. The weather
data contains weather observations from corresponding airport weather
stations on flight departure dates.

\subsection{Compiling and Cleaning}

\subsubsection{Flight Data}

I manually download
\textit{Bureau of Transportation Statistics' Airline On-Time Performance Data}
for January, March, September, and December in 2016, 2017, and 2018,
respectively. I import the data and compile using Pandas in Python (see
corresponding Jupyter NB). The resulting dataset has 5,851,068
observations and 21 features. To make the dataset manageable, I draw a
random subset (fraction=0.0015) from each month-year sample. The
resulting dataset contains 8,777 observations.

\subsubsection{Weather Data}

I use web-scraping methods in Python (see corresponding Jupyter NB) to
acquire historical weather data from \textit{Weather Underground}. I use
airport codes corresponding to origin airports for departures in the
flight data to scrape historical weather data from airport weather
stations. I acquire observations on temperature, precipitation, sea
level pressure, and max wind speed on the date of departure. The
resulting dataset contains 6,190 observations.

\subsubsection{Merged Data}

I merge the flight and weather data on the date of departure and origin
airport code. For the flight data, delays are identified as any flight
departing more than 15 minutes late: \texttt{DepDel15=1} if delayed and
\texttt{DepDel15=0} otherwise. Delays measured in minutes are given by
\texttt{DepDelay}. Since the supervised learning methods I utilise rely
on the assumption that \textbf{target variables} do not have missing
values, I drop observations if both \texttt{DepDel15} and
\texttt{DepDelay} are missing because such observations contain no
useful information for the analysis.

Since I am interested in predicting delays and delay lengths using
flight characteristics and weather observations, I consider the
proportion of missing values for these predictor variables. I find that
there are no missing observations in the flight data. However, around
70\% of observations have missing values for \texttt{Day.Average.Temp},
\texttt{High.Temp}, \texttt{Low.Temp}, \texttt{Max.Wind.Speed}, and
\texttt{Sea.Level.Pressure} (see Jupyter NB). Moreover, around 90\% of
observations have missing values for \texttt{Precipitation}. Dropping
observations missing weather data is costly in terms of observations.
Yet, the weather data is of interest in prediction and is likely
independent of many flight characteristics. I choose to omit
observations that have missing values for \texttt{Day.Average.Temp},
\texttt{High.Temp}, \texttt{Low.Temp}, \texttt{Max.Wind.Speed}, and
\texttt{Sea.Level.Pressure}.\\

Further omitting observations that have missing values for precipitation
may be discarding useful information: the correlation between
\texttt{DepDel15} and \texttt{Precipitation} (0.059) and between
\texttt{DepDelay} and \texttt{Precipitation} (0.026) are both
non-negligible, and; mean precipitation is higher for delayed departures
(0.446 inches) than for non-delayed departures (0.342 inches). Since
\texttt{Precipitation} is correlated with other weather observations, I
choose to impute missing values for \texttt{Precipitation} using
k-Nearest Neighbours on weather data. I optimise parameter \(k\) by
choosing \(k \in (0,100)\) to minimise the average MSE for out-of-sample
prediction across 50 random sub-samples of complete weather data (see
Jupyter NB). I impute missing values for \texttt{Precipitation} in the
merged dataset using \(k^*=2\). The resulting dataset has 2,693
observations and 27 features.\\

\subsection{Feature Engineering}

There is structure in the data that may be useful to exploit. For
instance, the data already splits flight dates into \texttt{Month},
\texttt{DayofMonth}, and \texttt{DayofWeek}, which may be relevant in
predicting flight delays if, for example, weekend flights are more prone
to delays. In a similar vein, I re-code \texttt{DepTimeBlk}, which gives
the astronomical time interval in which a departure is scheduled, as a
factor variable to be used in prediction. I likewise re-code
\texttt{ArrTimeBlk}. The variables \texttt{CRSDepTime} and
\texttt{CRSArrTime} give the scheduled departure and arrival times of a
flight in astronomical time. I find the difference in minutes between
scheduled departure and arrival times to categorise scheduled flight
lengths by hour in \texttt{SchFlTm}. The variable \texttt{Distance}
gives the flight distance in miles, I categorise flights by distance in
\texttt{DistGr} in intervals of 500 miles. I assign \texttt{InSt=1} if a
flight is within state and \texttt{InSt=0} otherwise using
\texttt{OriginStateName} and \texttt{DestStateName}. Finally, I use a
stricter definition of a departure delay than that of \texttt{DepDel15}.
I assign \texttt{Delayed=1} if \texttt{DepDelay>0} and
\texttt{Delayed=0} otherwise. The dataset used for the analysis contains
2,685 observations and 31
features\footnote{Note: I remove \texttt{Flights}, which gives the number of flights per flight journey, since it is equal to one for all observations and therefore contains no useful information.}.

\subsection{Summary}

The distribution of departure delay lengths
is\footnote{Early departures have negative delay lengths.}:\\

I consider the features that may be useful in distinguishing between
delayed and non-delayed flights. Delayed flights tend to have longer
scheduled flight times:\\

Accordingly, delayed flights tend to have greater flight distances:

\subsection{Methodology and Results}

To identify important predictors of flight delays, I use supervised
methods that are able to classify (i.e., predict) delayed and
non-delayed flights. I compare Logit, Lasso, Ridge, Elastic Net, Random
Forest, and Boosting methods. I consider the performance of these
methods and the top 5 most important predictors of flight delays they
each identify. I also consider the top 5 airlines, origins,
destinations, and top flight and weather features the methods
identify.\\

Since passengers likely wish to avoid costly delays, the performance of
the methods is evaluated using the true positive and false negative
rates of prediction. Accordingly, I cross-validate the fit of models
using the AUC as the performance metric. Moreover, since passengers
presumably wish to take their intended flight, I assess the methods
using false positive rates. Finally, I consider the accuracy of
prediction since a given passenger would wish to know whether a given
flight is likely to be delayed.\\

To fit the models, I divide the merged dataset into training (n=) and
test (n=) samples\footnote{Note: continuous variables are normalised.}.
I fit the models on the training sample and cross-validate, where
relevant, to tune model parameters. I then predict delays in the test
sample using parametised models.\\

First, I fit a Logit model that yields the following:\\

The logit model performs relatively poorly at predicting delays; for
instance, it assigns higher probabilities to delays over non-delays only
marginally better than a random guess. Perhaps this is due to
over-fitting on the training sample (i.e., the model delivers low bias
in-sample but high variance out-of-sample). I therefore turn to
shrinkage methods to strike a better bias-variance trade-off. In
particular, I fit and cross-validate Lasso, Ridge, and Elastic Net
models\footnote{Note: I tune shrinkage, $\lambda$ for Lasso (=), Ridge (=), and Elastic Net (= ). I tune sparsity for Elastic Net (= ).}
that yield:\\

Evidently, prediction performance improves using shrinkage methods. In
particular, Lasso, Ridge, and Elastic Net exhibit more desirable true
positive and false positive rates; as a result, the respective AUCs are
an improvement on logit. That said, false negative rates remain similar
to logit. The possibility of `creeping delays', caused by the
interaction of features that increase the probability of a delay,
suggest that non-linearities in the data may be an important aspect of
prediction. I therefore turn to non-linear methods. I grow and prune a
Random
Forest\footnote{Note: I manually tune to optimise the maximum number of features (=) and number of trees (=).}
and subsequently consider the predictive power of a coalition of
\textit{weak learners} using cross-validated Gradient-based Minimisation
(``GBM'')\footnote{Note: I manually tune to optimise the learning rate (=), number of trees (= ), and tree depth (=).},
which yield:\\

Non-linear models appear to offer no significant improvement on
shrinkage methods. Random Forest, in particular, delivers the worst rate
of prediction among the methods, despite having the best AUC. This is
perhaps due to over-fitting on the data. That GBM outperforms Random
Forest in prediction perhaps eludes to the fact that shrinkage methods
are able to exploit sparsity in the training data to aid in
prediction.\\

In summary, the fitted models yield:\\

\subsection{Conclusion}

The best performing method is Lasso. It delivers the highest true
positive rate (), the second-highest true negative rate (), and the
highest accuracy () among all methods. It identifies days of the month
as being the most important predictors of flight delays. This is in
contrast to Logit, Ridge, and Elastic Net, which select origins and
destinations, and Random Forest and GBM, which select weather and flight
features. That Lasso performs best in this context is unsurprising given
the multiplicity of dummy variables generated by the features in the
data. Many of the other models are likely over-fitting on the training
data; Lasso mitigates this issue by exploiting
sparsity\footnote{Note: that cross-validated Elastic Net chooses $\alpha$= speaks to this fact.}.
Evidently, predictions are noisy: the top 5 most relevant predictors by
category are relatively unstable between models. I use a rudimentary
distance
metric\footnote{For each top 5 category, I count the frequency of unique predictors given by the models. If all models agree, there would be 5 variables each occuring 5 times. I calculate the difference between the actual frequency of the predictors and 5. Then I sum the squared-difference to calculate the metric, which punishes top 5 categories twofold, for: (1) models that disagree with each other, and (2) the extent to which models disagree.}
to determine the stability of the predictions:

The most stable predictions across models are for which carriers are the
most important in determining delays. This is perhaps unsurprising given
the data presented above. A sensible choice for passengers might then be
to avoid airlines with a history of delays. Happy flying!

\end{document}
